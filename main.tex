\input{pre}
\usepackage{alltt}
\usepackage{bibunits}
\usepackage{enumitem}
\usepackage{ltablex}
\usepackage{booktabs}

\begin{document}
\thispagestyle{empty}
\begin{center}
\noindent 
\textbf{Правительство Российской Федерации \\Федеральное государственное автономное образовательное учреждение высшего образования}

\textbf{Национальный исследовательский университет}\\
\textbf{«Высшая школа экономики»}\\
\textbf{Факультет гуманитарных наук}\bigskip\\
\vfill
Образовательная программа \\
  «Фундаментальная и компьютерная лингвистика»\\
\vfill

\huge{ДИПЛОМНАЯ РАБОТА}\\
\large

На тему \\
\Large
«Профилирование локативного участника как основа лексических противопоставлений»\\
«Profiling of the locative participant as the basis of lexical oppositions»\\
\vfill
\vfill
\normalsize
\begin{flushright}
Студент 4 курса\\
группы №153 \\
Соколовский Матвей Ильич\bigskip\\
                       
Научный руководитель\\
канд. филологических наук, доц.\\
Резникова Татьяна Исидоровна\\

\end{flushright}
\vfill
\begin{center}
Москва --- 2019
\end{center}

\end{center}
\pagebreak
\tableofcontents
\pagebreak
\tab

\section{Введение} 
\subsection{О работе}
Идея данного исследования зародилась при взгляде на русские лексемы семантического поля "присваивание чужого" красть, грабить, воровать а также их ближайшие английские аналоги \textit{to steal и to rob}. Можно заметить что и в русском и в английском языках существуют отдельные лексемы, для описания той же ситуации профилируя локативного участника, т.е. выражая место в позиции прямого объекта. Следовательно, можно предположить что на уровне семантики ситуации есть причина для подобного профилирования.

При более подробном рассмотрении ситуация не так проста, потому как в обоих языках вместо локативного участникав позиции прямого объекта может также стоять и жертва, с которой место ассоциировано (\textit{грабить дворец} $\leftrightarrow$ \textit{грабить короля}) cм. раздел \ref{grabit}. Но во многих других глаголах присоединяющих локативного участника в позиции прямого дополнения замена его на посессора локативного участника невозможна: к примеру глагол \textit{подметать} может управлять либо локативным участником (\textit{подмести квартиру}), либо темой (\textit{подмести крошки}), но не одушевлённым посессором места. Таким образом область исследования ограничена контекстами в которых может встречаться локативный участник, а рассматривается в них то, как этот участник выражается при глаголе и с какими другими семантическими участниками он может чередоваться. В отношении способа выражения при глаголе можно выделить четыре ожидаемых стратегии:
\begin{enumerate*}[label=(\arabic*),itemjoin={\hskip2mm},after=\hskip2mm,before=\hskip2mm]
\item локативный участник может быть выражен в позиции прямого дополнения без изменения формы глагола (\textit{загружать песок в ящик --- загружать ящик песком}),
\item глагол может быть изменён таким образом, что локативный участник станет выражаться в позиции прямого дополнения (\textit{искать в комнате --- обыскать комнату}),
\item у глагола может быть неоднокоренной синоним управляющий локативным участником (\textit{красть --- грабить}),
\item локативный участник не может быть выражен в позиции прямого дополнения при описании ситуации (\textit{*раскупить магазин}).
\end{enumerate*}
В данной работе представлена попытка ответить на вопрос, насколько сильно семантика ситуации влияет на способ выражения локативного участника. Для этого мы анализируем модели управления глаголов описывающих ситуации с локативным участником в русском, и, сравниваем с некоторыми конструкциями из английского и немецкого языков. Идея применять методы московской семантической школы, собранные в статье Ю.Д. Апресяна \citep{апресян2005московской}, к типологическим данным упоминается впервые недавно, в работе Е.В,. Рахилиной и Т.И.Резниковой посвящённой фреймовому подходу к лексической типологии \citep{rakhilina2016frame}. Главный принцип, объединяющий формальную лексикографию с фреймовой лексической типологией звучит как \textit{"семантические
свойства лексики любого языка проявляются в её сочетаемости"}.

\subsection{Обзор литературы}
Настоящая работа примыкает сразу к нескольким направлениям лингвистического исследования. 

С одной стороны, в ней рассматриваются принципы лексикализации определенного концептуального содержания в типологической перспективе и тем самым она связана с активно развивающейся в последние десятилетия областью синхронной лингвистики - лексической типологией. Эта область восходит корнями к известной книге Брэнта Берлина и Пола Кая, посвящённой типологии цветообозначений \citep{berlin1969basic}, после чего лексической типологии посвящались уже более общие работы \citep{lehmann1990towards, koptjevskaja2008approaching, krv2015}. За полвека сложилось несколько основных подходов к её изучению:  \begin{enumerate*}[itemjoin={\hskip3mm},after=\hskip3mm,before=\hskip3mm]
    \item Традиционный психолингвистический метод использованный в исследовании Берлина и Кая до сих пор развивается типологами из института Макса Планка в Неймегене. Заключается он в том, что данные собираются с помощью опроса носителей, которым предоставляются экстралингвистические стимулы. Как за эти десятилетия был доработан традиционный метод и как он применяется сейчас можно увидеть в ряде публикаций Анфисы Маджид, из которых можно выделить исследования семантического поля глаголов "резать, рубить, ломать" \citep{majid2007semantic, majid2008cross}.
    \item Структуральный подход, предложенный Адрианой Лерер в её типологическом исследовании глагольного семнатического поля "приготовление пищи"  \citep{lehrer1969semantic}, на сегодняшний день считается устаревшим. Он заключается в описании глаголов методом компонентного анализа, где значение глагола состоит из независимых бинарных семантических показателей (напр. готовить с жиром или без жира, с паром или без пара и т.д.). Заполняя таблицу из глаголов и признаков можно обнаруживать межъязыковые соответствия.
    \item Подход представленный Анной Вержбицкой \citep{wierzbicka1972semantic, goddard1994semantic} основывается на теории универсального семантического метаязыка, набора семантических единиц, достаточного для описания любого концептуального явления, а следовательно, толкования любой лексемы любого языка. Несмотря на сложность толкования и значительную роль интроспекции исследователя при толковании, метод семантических примитивов и ныне применяется учениками Анны Вержбицкой. Примером его использования может служить исследование Клифа Годдарда посвящённое глаголам движения \citep{goddard1997semantics}.
    \item Последний и, на данный момент, наиболее перспективный подход разработан в рамках \hyperlink{mlext}{московсковской лексикотипологической группы} и подробно описан в статье Е.В. Рахилиной и Т.И. Резниковой \citep{rakhilina2016frame}. Он основывается на понятии семантического фрейма, прототипического значения, релевантного для лексической оппозиции. Фреймовая лексическая типология, как и структуральная типлогия Адрианы Лерер, оперирует с минимальными семантическими признаками, однако ключевое отличие в том, что здесь эти признаки не существуют независимо, а выявлять их помогает сравнение сочетаемости лексем, в соответствии с принципами московской семантической школы описанными в трудах Ю.Д. Апресяна  \citep{апресян1995избранные}.
\end{enumerate*}
С другой стороны, в данном исследовании рассматривается проблематика аргументной структуры глагола, а область семантических и синтаксических валентностей глагола традиционно не связана с лексической типологией. Аргументную структуру русских глаголов глубоко анализировали в ходе развития формальной лексикографии и разработки концептуальной модели "Cмысл $\Leftrightarrow$ Tекст"\footnote{теория Смысл$\Leftrightarrow$Текст была представлена Игорем Мельчуком в середине 70-ых годов и затем разрабатывалась им и рядом советских лингвистов. ТСТ представляет из себя многоуровневую систему, для построения моделей естественного языка. Подробнее см. \citep{мельчук1974опыт}}. Тот факт, что семантические и синтаксические аргументы глагола накладывают ограничения на его сочетаемость создаёт немало проблем для лексикографа. Основным путём решения этих задач стало предложение включать в толкование слова подробную информацию о семантическом классе глагола, его модели управления и сочетаемости. Обсуждение этой проблемы можно также найти в трудах Ю.Д. Апресяна \citep[119-156]{апресян1995избранные}, \citep[129-131]{апресян1995избранные2} и И.А. Мельчука \citep[134–139]{мельчук1974опыт}. Другой важной фигурой в системном анализе лексической семантики стала Е.В. Падучева, описавшая многие нетривиальные случаи моделей управления, среди которых, помимо прочего присутствуют и ситуации с локативными участниками \citep{падучева2004динамические, падучева2012неопределенно}. Ряд более конкретных проблем, напрямую сязанных с нашим исследованием, позже обсуждался в работах русских лингвистов: в статье Е.В. Муравенко  \citep{муравенко1998случаях} идёт речь о случаях нетривального соответствия семантических и синтаксических валентностей а Л.Л. Цинман и В.Г. Сизов \citep{цинман1998модель} в своей работе предлагают решения для формального анализа модели управления. 
\par Помимо русских лингвистов общий концепт аргументной структуры глагола и типологический взгляд на явление неплохо описан в статьях Мартина Хаспельмата \citep{haspelmath2004valency, haspelmath2015comparing}. А противопоставление семантических аргументов семантическим адъюнктам (\textit{актанты VS сирконстанты}) по принципу обязательности поверхностного выражения было введено ещё в классической книге Люсьена Тесньера \citep{tesniere1959elements}.

\par В заключение можно упомянуть работы, тем или иным образом напрямую затрагивающие предмет нашего исследования. Противопоставление глаголов \textit{to rob} и \textit{to steal}, которое положило начало настоящему исследованию, рассматривалось ранее в   \citep[45-48]{goldberg1995constructions}, \citep{thorgren2005transaction} and \citep{van2007role}. В русском языке смеантическое поле 'присваивание чужого' насчитывает больше лексикализаций, и ему посвящён ряд работ Ю.М. Муратова \citep{муратов2014красться, муратовгенетическая}, однако анализ с точки зрения аргументной структуры и модели управления у Ю.М.Муратова не представлен. Типология семантического поля "поиск", к которому относятся рассматриваемые нами глаголы \textit{искать} и \textit{обыскать}, имеющие семантического локативного участника, подробно исследована и описана в статьях сборника 'ЕВРика!' \citep{еврика}.

\section{Семантическое поле: <<присваивание чужого>>} \label{grabit}
Данное исследование началось с анализа фрейма <<присваивание чужого>>, а именно русских глаголов \textit{красть, грабить, воровать} и английских \textit{to rob, to steal}. На первый взгляд можно сказать, что русские глаголы \textit{красть} и \textit{грабить} описывают одну и ту же ситуацию, с одним и тем же набором участников. А ключевая разница между ними заключается в модели управления. Если у \textit{красть} в позиции прямого объекта находится объект (\textit{красть деньги}), то у \textit{грабить} в этой позиции оказывается место (\textit{грабить магазин}), а объект утрачивает способность к поверхностному выражению. 

\par Однако при более подробном рассмотрении выясняется, что дело обстоит сложнее. В таблицах \ref{mukrasti} и \ref{mugrabiti} представлены модели управления этих двух глаголов. 


\begin{table}[H]
\vspace{+10pt}
\centering
\parbox{.49\linewidth}{
\centering
\begin{tabular}{c|c|c|c}
\hline
\textsc{агенс}&\textsc{жертва}&\textsc{тема}&\textsc{место}\\
\hline
NOM& у + GEN& ACC & из + GEN\\
\hline

\end{tabular}
\caption{красть}
\label{mukrasti}
}
\parbox{.49\linewidth}{
\centering
\begin{tabular}{c|c}
\hline
\textsc{агенс}&\textsc{жертва/место}\\
\hline
NOM&ACC\\
\hline
\end{tabular}
\caption{грабить}
\label{mugrabiti}
}
\end{table}

В таблице \ref{mukrasti} показано, что актанты жертва и место разделяются и могут выражаться одновременно (\textit{У \textbf{меня} из \textbf{кармана} украли кошёлек; Злоумышленники украли у \textbf{миллионера} из \textbf{квартиры} чемодан с деньгами}). При глаголе \textit{грабить} тема не выражается, а в позиции прямого объекта может быть как место, так и жертва. Ввиду того, что жертва --- человек, она обладает большей пациентивностью. Следовательно она с большей вероятностью и есть прототипический объект при глаголе \textit{грабить}, что в некоторой степени подтверждают данные НКРЯ: локативный участник занимает позицию прямого дополнения примерно в 40\percent случаев. Появление в этой позиции места часто может объясняться метонимическим переносом с жертвы.

\par Таким образом, о противопоставлении \textit{воровать -- грабить} нельзя сказать, что именно профилирование локативного участника лежит в его основе.

\section[Локативный участник '\textsc{мишень}']{Локативный участник '\textsc{мишень}'\footnote{Не выделяется в отдельную семантическую валентность в работах Ю.Д. Апресяна и И.А. Мельчука \citep{апресян1995избранные2, mel2004actants},  а у Е.В. Падучевой в \citep{падучева2004динамические} мишенью называется не то же что у нас. Здесь \textsc{мишень} полностью соответствует тому, что во \hyperlink{fn}{FrameNet} называется \textsc{target}}}

Ситуаций включающих участника \textsc{мишень} не так много. Во \hyperlink{fn}{FrameNet} мишень (\textsc{target}) противопоставляется цели (\textsc{goal}), на том основании, что в ситуациях с участником мишенью в центре стоит дистанционное взаимодействие агенса и локативного участника, а в ситуациях с целью --- перемещение темы в локативного участника.

\lb{a}{Мальчик попал [в скворечник \textsc{target}] камнем.}
\lb{b}{Мальчик кинул камень [в скворечник \textsc{goal}].}

Если в примере \rf{b} камень --- перемещаемая \textsc{тема}, которая обязана быть выражена, то в \rf{a} камень --- \textsc{инструмент} и может быть опущен. \textsc{мишень} также появляется при глаголах \textit{целиться, метить, стрелять, попадать, промахиваться, угодить и т.д.}. Обязательность выражения мишени в примере \rf{a} уже говорит о её важности для ситуации, однако это не единственный аргумент. Русский язык позволяет отразить это на синтаксическом уровне выражая мишень в позиции прямого объекта при глаголах \textit{расстрелять, подстрелить, поразить, подбить/сбить}, а также в конструкции \textit{взять на мушку/прицел}. Таким образом практически любая ситуация включающая в себя партиципанта 'мишень' может быть описана таким образом, что мишень окажется в позиции прямого дополнения.

По поводу глагола промахнуться известно немало споров \citep[148]{апресян1995избранные},\citep[135]{мельчук1974опыт}. Ему же посвящена немалая часть второй главы Лингвистики конструкций \citep[315-317]{рахилина2010лингвистика}, однако современные данные \hyperlink{rusc}{НКРЯ} существенно дополняют. Считается, что глагол \textit{промахнуться}, означающий \textit{'не попасть в цель'}, наследует семантические валентности глагола \textit{попасть}, однако не может выразить их синтаксически. И.А. Мельчук \citep{mel2004actants} объясняет это тем, что у этого глагола есть пресуппозиция самого факта стрельбы, который в свою очередь включает в себя всех участников (инструмент, средство, цель/мишень). Ю.Д. Апресян называет "контексты с противопоставительным или уступительным значением" единственным исключением, т.е. случаем когда семантические валентности отличные от субъекта выражаются:

\lb{medved}{На расстоянии всего пяти метров он промахнулся в медведя из великолепного бельгийского ружья центрального боя.}

Примеры с конструкцией \textit{промахнуться в Х} из примера \rf{medved} действительно достаточно маргинальны, однако на это влияет семантика предлога 'в' описывающего движение в мишень как при глаголе попасть. Глагол промахнуться описывает ситуацию, в которой средство проходит мимо мишени, соответственно, если выражать мишень при глаголе, то предлог 'мимо' более ожидаем. И это подтверждается: количество примеров употреблений конструкции \textit{промахнуться мимо Х-а} увеличивается:

\lb{mimo}{Стоило солдату промахнуться мимо урны с окурками ― баран уже жевал бычок.\\ \textcolor{gray}{[Андрей Колесников. Отряд (1997) // «Столица», 1997.06.10]}}

\lb{mimo2}{Я пошел лесной дорогой и промахнулся мимо деревни. \\ \textcolor{gray}{[Владимир Шаров. Воскрешение Лазаря (1997-2002)]}}

\lb{mimo3}{От омерзения Глеба передёрнуло, и он промахнулся пальцем мимо клавиши \\ стеклоподъёмника. \textcolor{gray}{[Алексей Иванов. Комьюнити (2012)]}}

Учащение подобных примеров говорит о том, что конструкция промахнуться мимо Х-а становится всё более грамматична, что также свидетельтвует о важности мишени в данном семантическом поле.

%Переходя к рассмотрению 
%%%
\section{Периферийные локативные участники}
\subsection{\textsc{место -- инструмент}}
Рассмотрим следующие примеры:
\lb{ins}{а. Вася говорит с Петей по телефону.\\
b. Вася общается по телефону.\\
c. Вася говорит с Петей в скайпе / по скайпу.\\
d. Вася общается в скайпе / $^?$по скайпу.\\
e. Вася отправил сообщение в мессенджере / *по мессенджеру.}


\subsection{Расщеплённая валентность. \textsc{место -- пациенс}}
Термин "расщепление валентности" вводит Ю.Д. Апресян \citep[153]{апресян1995избранные} для описания случаев выражения одной семантической валентности двумя синтаксическими. Например, в первом предложении примера \rf{rasval} 

\lb{rasval}{Вася ударил по Петиной руке.\\ Вася ударил Петю по руке.}


объектная валентность \textit{рука Пети} расщепляется на два синтаксических аргумента. Большой вопрос заключается в том, какие семантические роли выражают новые участники \textit{рука} и \textit{Петя}. С одной стороны можно считать, что \textit{рука} это \textsc{пациенс}, а \textit{Петя} --- \textsc{посессор} и ввиду неодушевлённости пациенса и одушевлённости посессора именно последний тяготеет к позиции прямого объекта. С другой стороны, можно воспринимать \textit{Петю} как \textsc{пациенс} и тогда \textit{рука} --- это \textsc{место}.

\section{Русские приставки \textit{об-} и \textit{раз-}}

В русском языке локативный участник ситуации чаще всего появляется в позиции прямого дополнения при глаголах образованных при помощи приставок с семантикой предельности.
\lb{isk}{Иван искал кошку в комнате.\\ Иван \textbf{об}ыскал комнату \textcolor{gray}{в поисках кошки}.}
\lb{isk2}{Иван рисовал котов на стене.\\Иван \textbf{раз}рисовал стену \textcolor{gray}{котами}.}

В этих случая приставка с семантикой предельности выносит на первый план охват всего пространства целиком, а соответственно повышает его значимость в описываемой ситуации. 


\section{Перспективы автоматизации исследования}


\section{Заключение} \label{final}


\pagebreak
\section{Литература}
\renewcommand{\bibsection}{}
\bibliography{bibliography.bib} 

\section{Онлайн ресурсы}





%\noindent \hypertarget{tenten}{Корпуса ruTenTen$11$, deTenTen$13$, enTenTen$15$ и BNC  на базе Sketch Engine}:\\\url{https://app.sketchengine.eu} \medskip

%\noindent \hypertarget{glosbe}{Многоязычный онлайн словарь Glosbe}:\;\url{https://ru.glosbe.com} \medskip

\noindent \hypertarget{rusc}{Национальный корпус русского языка (НКРЯ)}:\;\url{http://ruscorpora.ru/} \medskip

%\noindent \hypertarget{gt}{Переводчик Google}:\; \url{https://translate.google.com} \medskip

%\noindent \hypertarget{yt}{Переводчик Yandex}:\; \url{https://translate.yandex.com} \medskip

\noindent \hypertarget{mlext}{Проект 'московская лексикотипологическая группа' (MLexT)}:\; \url{http://lextyp.org/} \medskip

%\noindent \hypertarget{rusvectores}{Семантические модели для русского языка RusVectōrēs}:\;\url{https://rusvectores.org/ru} \medskip

%\noindent Словари и энциклопедии на Академике:\; \url{https://dic.academic.ru} \medskip
\noindent \hypertarget{fn}{Семантическая база FrameNet:}\; \url{https://framenet.icsi.berkeley.edu/fndrupal/} \medskip

\noindent \hypertarget{valpal}{Типологическая база моделей управления ValPaL}\; \url{http://www.valpal.info/} \medskip

\end{document}