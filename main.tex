\input{pre}
\usepackage{alltt}
\usepackage{bibunits}
\usepackage{enumitem}
\usepackage{ltablex,booktabs}
\begin{document}
\definecolor{zzttqq}{rgb}{0.26666666666666666,0.26666666666666666,0.26666666666666666}
\definecolor{cqcqcq}{rgb}{0.7529411764705882,0.7529411764705882,0.7529411764705882}
\thispagestyle{empty}
\begin{center}
\noindent 
\textbf{Правительство Российской Федерации \\Федеральное государственное автономное образовательное учреждение высшего образования}

\textbf{Национальный исследовательский университет}\\
\textbf{«Высшая школа экономики»}\\
\textbf{Факультет гуманитарных наук}\bigskip\\
\vfill
Образовательная программа \\
  «Фундаментальная и компьютерная лингвистика»\\
\vfill

\huge{ДИПЛОМНАЯ РАБОТА}\\
\large

На тему \\
\Large
«Профилирование локативного участника как основа лексических противопоставлений»\\
«Profiling of the locative participant as the basis of lexical oppositions»\\
\vfill
\vfill
\normalsize
\begin{flushright}
Студент 4 курса\\
группы №153 \\
Соколовский Матвей Ильич\bigskip\\
                       
Научный руководитель\\
канд. филологических наук, доц.\\
Резникова Татьяна Исидоровна\\

\end{flushright}
\vfill
\begin{center}
Москва --- 2019
\end{center}

\end{center}
\pagebreak
{
  \hypersetup{linkcolor=black}
  \tableofcontents
}
\pagebreak

\section{Введение} 
Локативный участник ситуации затрагивается в большом количестве работ по семантике. 
\citep{еврика}
\citep{падучева2004динамические}

\section{Заключение} \label{final}


\pagebreak
\section{Литература}
\renewcommand{\bibsection}{}
\bibliography{bibliography.bib} 

\section{Онлайн ресурсы}





%\noindent \hypertarget{tenten}{Корпуса ruTenTen$11$, deTenTen$13$, enTenTen$15$ и BNC  на базе Sketch Engine}:\\\url{https://app.sketchengine.eu} \medskip

%\noindent \hypertarget{glosbe}{Многоязычный онлайн словарь Glosbe}:\;\url{https://ru.glosbe.com} \medskip

%\noindent \hypertarget{ruscorpora}{Национальный корпус русского языка (НКРЯ)}:\;\url{http://ruscorpora.ru} \medskip

%\noindent \hypertarget{gt}{Переводчик Google}:\; \url{https://translate.google.com} \medskip

%\noindent \hypertarget{yt}{Переводчик Yandex}:\; \url{https://translate.yandex.com} \medskip

%\noindent \hypertarget{mlext}{Проект московская лексикотипологическая группа (MLexT)}:\; \url{http://lextyp.org} \medskip

%\noindent \hypertarget{rusvectores}{Семантические модели для русского языка RusVectōrēs}:\;\url{https://rusvectores.org/ru} \medskip

%\noindent Словари и энциклопедии на Академике:\; \url{https://dic.academic.ru} \medskip


\end{document}